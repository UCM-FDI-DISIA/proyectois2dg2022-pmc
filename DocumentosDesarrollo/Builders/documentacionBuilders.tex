\documentclass{article}
\title{Funcionamiento de los GameBuilders}
\date{\today}
\author{Grupo PMC}
\usepackage[spanish]{babel}
\usepackage[top=2.5cm, bottom=2.5cm, left=3cm, right=3cm]{geometry}
\usepackage{listings}
\clubpenalty=10000 %líneas viudas NO
\widowpenalty=10000 %líneas viudas NO

\begin{document}
\maketitle
\subsection*{Clase GameBuilder}
La clase abstracta \textit{GameBuilder} se encarga de gestionar la creación de un \textit{Game} a través de un JSON. De esta forma, independientemente de si queremos cargar una partida a partir de un fichero, de la consola o de cualquier otro lugar, podemos generar un juego.

Para crear el \textit{Game} tenemos un  método llamado createGame(JSON game) que recibe un \textit{Game} en formato JSON y devuelve un objeto \textit{Game}. Esta función lee del JSON el modo de juego (que es un String) y llama a un método parse(String gameMode) que devuelve una instancia del GameBuilder del tipo de juego que se pretende crear.

Después se llama al método abstracto generateGame(JSON game), que devuelve el Game contenido en el JSON y que implementan cada una de las clases hijas, puesto que en cada modo de juego tiene un JSON distinto.

\subsubsection*{GameBuilderClassic y GameBuilderTeams}
Estas clase extienden de \textit{GameBuilder} y surgen de la urge la necesidad de crear una clase hija por cada tipo de juego, que sepa como interpretar el JSON y hacer ``new''. En otras palabras, implementan el metodo generateGame(JSON game).

Para que funcione correctamente el método parse de la superclase necesitan tener un método matchBuilder(String gameMode), que recibe el nombre del modo de juego y devuelve true si coincide con el tipo de juego que sabe generar.

\subsection*{Método createGame() sin parámetros}
Se sobrecargará el método createGame para que exista una version sin parámetros. Si no se recibe un JSON, la clase GameBuilder interpretará que tiene que preguntar por consola los datos necesarios para generar un juego. Para ello, preguntará por el modo de juego al que se desea jugar, lo parseará al GameBuilder correspondiente y llamará al método askGame() que implementan cada una de las clases hijas. Este método devuelve un JSON que contiene los datos del Game que se acaban de recopilar. Después, se llamará al método createGame(JSON game) con el JSON que acabamos de generar.




\end{document}