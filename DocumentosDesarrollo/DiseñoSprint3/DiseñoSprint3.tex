\documentclass{article}
\title{DISEÑO DEL SPRINT 3}
\date{\today}
\author{Grupo PMC}
\usepackage[spanish]{babel}
\usepackage[top=2.5cm, bottom=2.5cm, left=3cm, right=3cm]{geometry}
\usepackage{listings}
\clubpenalty=10000 %líneas viudas NO
\widowpenalty=10000 %líneas viudas NO

\begin{document}
\maketitle

\subsubsection*{Board}
Vamos a dejar más determinada la capacidad del usuario de elegir las características del juego. Ahora va a haber sólo 3 tamaños de tablero: pequeño, mediano o grande y van a ser específicos de cada forma. También habrá un conjunto determinado de formas entre las que elegir e internamente crearemos el tablero con la forma y el tamaño determinados.

Del mismo modo, la selección del tablero y del tamaño determinará el número máximo de jugadores posibles, para que el juego sea plausible en el tablero seleccionado.

Una buena forma de tener determinadas y marcadas todas estas cosas es tener los tableros ya dibujados y determinados en unos ficheros ``.txt'' desde los que se puede cargar toda la información relativa al mismo.

\subsubsection*{SaveLoadManager}
Los ficheros de guardado ahora tienen nombre propio. Cuando guardamos una partida, creamos un fichero con ese nombre y guardamos el nombre en un fichero de nombres, para que luego al buscar para cargar partida el juego sepa decirte todas las partidas guardadas que hay.

Para poder repetir las partidas, va a haber que guardar durante la partida el estado del juego en cada turno en un fichero que llamaremos \textit{currentGame.txt}. Al final, se dedice si se quiere guardar la replay de la partida actual para verla más tarde.

Las replays tienen que poder dar un paso hacia delante, hacia atrás y mantenerse en pausa viendo un paso concreto de la partida y deben mostrar la salida por pantalla que se ve durante la partida tal y como estaba en ese momento.

Para gestionar esta nueva funcionalidad se creará una nueva clase Replay que irá construyendo cada estado a partir del \textit{Game}. Debería ser el SaveLoadManager quien se encargue de guardar y cargar los replays completos en un fichero a través de los métodos \textit{saveReplay()} y \textit{loadReplay()}.

\subsubsection*{Game}
Para implementar las funcionalidades de distintos juegos, la clase \textit{Game} va a pasar a ser una clase abstracta. Las clases de cada modo de juego será una extensión de dicha clase que encapsulará de forma local lo que se espera de ese modo.

Por el momento, para hacer la clase de juego por equipos lo único que cambia es la forma de imprimir el ranking y a lo sumo dos nuevos atributos que sumen las puntuaciones de los jugadores de cada equipo y que sean la puntuación total de los diferentes equipos. Puedes comerte los cubos de tus compañeros aunque las puntuaciones no cambian.



\end{document}