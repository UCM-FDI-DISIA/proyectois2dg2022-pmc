\documentclass{article}
\begin{document}
\title{Arquitectura}
\maketitle

\section{Patrón MVC}

\section{Clases}
Las clases del proyecto con las siguientes:

\textbf{GameBuilder}: Builder genérico de Game. Se encarga de generar GameClassic o GameTeams con sendas llamadas a GameClassicBuilder y GameTeamsBuilder.

\textbf{GameClassicBuilder}: Builder de GameClassic.

\textbf{GameTeamsBuilder}: Builder de GameTeams.

\textbf{Client}: Clase intermediara entre el ClientController y la vista.

\textbf{Command}: Clase genérica de comando que parsea los comandos introducidos para poder llamas a las clases específicas que gestionan dichos comandos.

\textbf{ExitCommand}: Clase que, dado el comando exit, termina el juego.

\textbf{HelpCommand}: Clase que, dado el comando help, muestra los comandos disponibles a ejecutar.

\textbf{PlaceCubeCommand}: Clase que, dado el comando de poner cubo, pide a Game que se coloque un cubo en la posición indicada.

\textbf{SaveCommand}: Clase que, dado el comando save, llama a SaveLoadManager para que guarde la partida en el instante en el que se llama al comando.

\textbf{Controller}: El controlados (en el sentido del MVC), intermediario de la vista y el modelo.

\textbf{SaveLoadManager}: Clase encargada de la carga, guardado y borrado de partidas y replays.

\textbf{Board}: Clase que define la estructura y lógica del tablero del juego.

\textbf{Color}: Clase de enumerados con métodos auxiliares para la representación de los colores.

\textbf{Cube}: Clase que representa un cubo, con su posición y jugador al que pertenece.

\textbf{Game}: clase abstracta que representa una visión general del juego, que luego se particulariza en GameClassic y GameTeams. Es un modelo (en el sentido del MVC).

\textbf{GameClassic}: Clase que implementa la lógica del juego en el modo Classic (jugadores individuales).

\textbf{Gameteams}: Clase que implementa la lógica del juego en el modo Teams (jugadores por equipos).

\textbf{Player}: Clase que representa a un jugador humano o IA y sus características (color, nombre, puntuación...).

\textbf{Replayable}: Interfaz que extiende de Reportable y que incluye un método toString() con el fin de encapsular Game para generar estados, que son usados por multitud de clases.

\textbf{Reportable}: Interfaz a través de la cual, implementada en clases, permite ejecutar un rport(), serialización del objeto en formato .json a través de unos estándares definidos.

\textbf{Rival}: Interfaz que permite unificar si se juega contra un jugador o contra un equipo mediante el concepto de rival.

\textbf{Shape}: Clase de enumerados con métodos auxiliares para la representación de las formas del tablero.

\textbf{Team}: Clase que representa a un equipo con el modo de juego GameTeams.

\textbf{TurnManager}: Clase que gestiona el paso de turnos.

\textbf{GameState}: Clase que representa el estado del juego en un momento determinado.

\textbf{Replay}: Clase a través de la cual se gestionan las replays, pudiendo por ejemplo avanzar y retroceder en las mismas.

\textbf{Rolit}: Clase de punto de partida de la aplicación. En esta se muestra el me´nu principal y se decide si acceder al modo GUI o al modo consola.

\textbf{HelpException}: Clase que representa la excepción ejecutada desde el comando HelpCommand.

\textbf{Server}: Clase que gestiona los clientes desde la perspectiva del servidos y que procesa los mensajes emitidos desde los clientes.

\textbf{ServerClient}: Clase que representa cada cliente desde la perspectiva del servidor.

\textbf{ServerClientThread}: Clase que interactúa con el cliente enviando y recibiendo mensajes.

\textbf{ServerView}: Clase que abre el diálogo a través del cuál el usuario pude abrir y detener un servidor.

\textbf{WaitPlayerThread}: Thread a través del cual el servidor va recibiendo las coneixones entrantes de los distintos clientes que se van conectando.

\textbf{GreedyStrategy} Estrategia IA consistente en una búsqueda superficial con el algoritmo Minimax.

\textbf{MinimaxStrategy}: Estrategia IA consistente una búsqueda en profundidad con el algoritmo Minimax y la poda alfa-beta.

\textbf{RandomStrategy}: Estrategia IA consistente en introducir un cubo de forma aleatoria siempre que este esté en una posición válida.


\textbf{SimplifiedBoard}: Encapsulación del tablero a través del cual la IA decide sus movimientos.

\textbf{Strategy}: Clase genérica estrategia IA de la cual heredan las estrategias específicas.

\textbf{Pair}: Clase que representa un par, utilidad básica para varias partes del código.

\textbf{StringUtils}: Clase que recopila distintas utilidades para trabajar con cadenas, básico para varias partes del código.

\textbf{RolitObserver}: Interfaz que declara las notificaciones a través de las cuales se trabaja en el patrón observador implementado para la estructura modelo-vista-controlador para particularidades del juego.

\textbf{ConsoleWindow}: Interfaz de ventana genérica para el modo consola

\textbf{DeleteGameWindow}: Ventana para la consola a través de la cual se puede borrar un juego guardado.

\textbf{LoadGameWindow}: Ventana para la consola a través de la cual se puede cargar un juego guardado.

\textbf{LoadReplayWindow}: Ventana para la consola a través de la cual se puede cargar un replay.

\textbf{MainBashWindow}: Ventana para la consola a través de la cual se muestra el menú principal.

\textbf{NewGameClassicWindow}: Ventana para la consola a través de la cual se crea un nuevo juego con modo GameClassic.

\textbf{NewGameTeamsWindow}: Ventana para la consola a través de la cual se crea un nuevo juego con modo GameTeams.

\textbf{NewGameWindow}: Ventana para la consola a través de la cual se crea un nuevo juego.

\textbf{PlayWindow}: Ventana para la consola a través de la cual se muestra el hilo de ejecución de una partida.

\textbf{SaveReplayWindow}: Ventana para la consola a través de la cual se puede guardar la partida en el estado actual.

\textbf{BoardGUI}: Clase que gestiona la representación del board en modo GUI

\textbf{BoardRenderer}: Renderer Swing que diseña los ComboBox que muestran las distintas formas de tablero que el usuario puede escoger.

\textbf{CeldaGUI}: Clase que representa una celda del tablero en modo GUI, responsabilizándose de su diseño y de recoger los clic del usuario.

\textbf{ChooseTeamFromServerDialog}: Diálogo GUI que, desde el cliente, se escoge en qué equipo desea unirse el usuario a la hora de conectarse a una partida en red con modo GameTeams.

\textbf{ColorRenderer}: Renderer Swing que diseña los ComboBox que muestran los distintos colores el usuario puede escoger.

\textbf{ControlPanel}: Panel Swing que muestra, según el caso, guardar partidas y el avance y retroceso en los replays. Se coloca en la parte superior de la ventana de la aplicación.

\textbf{DeleteGameDialog}: Diálogo GUI a través del cual el usuario puede eliminar partidas guardadas.

\textbf{JoinServerDialog}: Diálogo GUI a través del cual el cliente puede conectarse a un servidor introduciendo los datos de este.

\textbf{LoadFileDialog}: Diálogo GUI a través del cual el usuario puede cargar partidas guardadas.

\textbf{MainWindow}: Ventana principal del modo GUI sobre la cuál se colocan todos los componentes y nacen todos los diálogos.

\textbf{Observable}: Interfaz que declara los métodos de añadir y quitar observador para la implementación del patrón Observador.

\textbf{RankingTableModel}: Tabla Swing (GUI) que muestra la tabla de las puntuaciones en tiempo real de los usuarios de una partida.

\textbf{ReplayObserver}: Interfaz que declara las notificaciones a través de las cuales se trabaja en el patrón observador implementado para la estructura modelo-vista-controlador para particularidades del Replay.

\textbf{SaveReplayDialog}: Diálogo GUI a través del cual el usuario puede guardar la repetición de una partida finalizada.

\textbf{StatusBar}: Componente GUI situado en la parte inferior de la aplicación que muestra las notificaciones lanzadas por el modelo.

\textbf{TurnAndRankingBar}: Panel Swing (GUI) que junta tanto el ranking como el turno del usuario actual en la partida.

\textbf{CreateGameDialog}: Diálogo GUI a través del cual el usuario puede crear una partida.

\textbf{CreateGameWithPlayersDialog}: Clase que se encarga de los jugadores dentro del CreateGameDialog (GUI).

\textbf{CreatePlayersPanel}: Panel Swing (GUI) encargado de gestionar los jugadores introducidos en el CreateGameDialog.

\textbf{CreateTeamsPanel}: Panel Swing (GUI) encargado de gestionar los equipos introducidos en el CreateGameDialog.

\textbf{GameConfigurationPanel}: Panel Swing (GUI) que recoge la configuración de juego especificada en el CreateGameDialog.

\textbf{PlayerDataPanel}: Panel Swing (GUI) que recoge los datos introducidos de jugadores especificados en el CreateGameDialog.

\textbf{TeamDataPanel}: Panel Swing (GUI) que recoge los datos introducidos de equipos especificados en el CreateGameDialog.

\textbf{RolitBorder}: Extensión de Border (Swing, GUI) adaptada para el diseño particular decidido para la aplicación.

\textbf{RolitButton}: Extensión de JButton (Swing, GUI) adaptada para el diseño particular decidido para la aplicación.

\textbf{RolitCheckBox}: Extensión de JCheckBox (Swing, GUI) adaptada para el diseño particular decidido para la aplicación.

\textbf{RolitComboBox}: Extensión de JComboBox (Swing, GUI) adaptada para el diseño particular decidido para la aplicación.

\textbf{RolitIconButton}: Extensión de JButton con icono (Swing, GUI) adaptada para el diseño particular decidido para la aplicación.

\textbf{RolitPanel}: Extensión de JPanel (Swing, GUI) adaptada para el diseño particular decidido para la aplicación.

\textbf{RolitRadioButton}: Extensión de JRadioButton (Swing, GUI) adaptada para el diseño particular decidido para la aplicación.

\textbf{RolitTextArea}: Extensión de JTextArea (Swing, GUI) adaptada para el diseño particular decidido para la aplicación.

\textbf{RolitTextField}: Extensión de JTextField (Swing, GUI) adaptada para el diseño particular decidido para la aplicación.

\textbf{RolitToolBar}: Extensión de JToolBar (Swing, GUI) adaptada para el diseño particular decidido para la aplicación.


\end{document}