\documentclass{article}
\begin{document}
\title{Documento trabajo individual}
\maketitle
\section*{Sergio Miguel García Jiménez}
\subsection*{Código}

\begin{itemize}
\item Crear parte del menú principal en modo consola.
\item Colaborar en el diseño de la clase Player.
\item Desarrollar gran parte de la lógica de cargar, guardar y borrar partida (SaveLoadManager).
\item Contribuir a que modelo trabaje exclusivamente con JSON. Esto incluía la concepción del diseño de reports así como su implementación en las distintas clases, tarea inicialmente desarrollada por Leo, Virginia y yo, y continuada por el resto de compañeros.
\item Desarrollé junto con Leo toda la funcionalidad de red, incluyendo elementos de la GUI que precisa, su extensión a todos los modos de juego y la lógica detrás de las conexiones.
\item Me encargué junto a otros del desarrollo de la estructura de la GUI; concretamente en lo referente al tablero, diálogos de cargar, crear, borrar partida y la interacción de la GUI con el modelo, así como una participación activa en su pulido y debug por las posteriores características introducidas por el modelo.
\item Realicé diversas labores de optimización para solventar un pobre rendimiento de la GUI a la hora de mostrar el tablero, actualizándose ahora en tiempo casi instantáneo y constante.
\item El debug de las características que estaba desarrollando conllevó a debugear otras clases referentes al modelo para perfeccionarlo.
\item Encargado de realizar algunos merge y etiquetado.
\item Proponer y solventar algunas de las issues.
\end{itemize}

\subsection*{Documentación}
\begin{itemize}
\item He realizado los primeros diagramas de clases y UML para ofrecer unos primeros prototipos para poder determinar por qué estilo de diseño queríamos adoptar.
\item Concebí el diseño de la refactorización de las historias de usuario para adaptarlas al formato de la asignatura, tarea después completada por compañeros.
\item Creación de los primeros artículos y la estructura preliminar de la Wiki en los primeros sprints.
\item He realizado los primeros diagramas de clases y UML para ofrecer unos primeros prototipos para poder determinar por qué estilo de diseño queríamos adoptar.
\item Trabajo UML preliminar con Modelio.
\item Pugnar por pasar de trabajar en el software Modelio a PlantUML para trabajar con mayor comodidad.
\end{itemize}

\section*{Leonardo Macías Sánchez}
\subsection*{Código}
\begin{itemize}
\item Concepción, diseño y desarrollo de la clase SaveLoadManager, que en sprints futuros sería refactorizada por Sergio para adaptarla a las nuevas necesidades.

\item Desarrollo de las clases Player, Cube y Color, sobre todo durante la etapa inicial del proyecto.

\item Refactorización de la clase Board para que contuviera la lógica del juego.

\item Desarrollo y adaptación del juego a la red, incluyendo una extensión de la interfaz gráfica que facilitase el funcionamiento al usuario. Esta tarea fue desarrollada conjuntamente con Sergio.

\item Diseño y desarrollo del paquete replay, que abarca la interfaz Replayable y las clases Replay y GameState. Además de implementar toda la lógica, también me he dedicado a su correcta visualización, tanto en consola como en GUI.

\item Principal impulsor y defensor del uso de JSONObjects en el proyecto. Ideé la interfaz Reportable y me encargué del diseño de los JSON y de su posterior implementación junto con Sergio y Virginia.

\item Diseño e implementación de los métodos toString() del modelo.

\item Colaboración en el diseño de los GameBuilders junto con Juan Diego y Daniel.

\item Refactorización, junto a Juan Diego, de la clase Game para poder introducir las modalidades de GameTeams y GameClassic.

\item Ayudar a Virginia y Mar con problemas que surgieron al crear JUnits.

\item Refactorización completa de todas las clases de la GUI, rediseñandola al completo (a excepción del tablero) y creando nuestros propios componentes visuales, los RolitComponents.

\item Propuse la creación de la clase TurnManager.

\item Labores de debug.

\end{itemize}

\subsection*{Documentación}
\begin{itemize}
\item Diseño de los reports.
\item Redactar el funcionamiento de los Builders.
\item Desarrollo de diagramas de secuencia.
\item Redactar algunos de los sprint reviews y retrospectives.
\item Colaboración en el desarrollo de la wiki.
\item Búsqueda de una alternativa a Modelio para el desarrollo de diagramas de clases.
\end{itemize}

\end{document}