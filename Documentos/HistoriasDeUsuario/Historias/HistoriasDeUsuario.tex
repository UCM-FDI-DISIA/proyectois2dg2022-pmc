\documentclass[../../FINAL/Scrum/SCRUM.tex]{subfiles}
\title{HISTORIAS DE USUARIO}
\date{\today}
\author{Grupo PMC}
\usepackage[spanish]{babel}
\usepackage{soul}
\usepackage{ulem}
\usepackage[top=2.5cm, bottom=2.5cm, left=3cm, right=3cm]{geometry}
\clubpenalty=10000 %líneas viudas NO
\widowpenalty=10000 %líneas viudas NO

\begin{document}
\newcounter{epic}
\newcounter{story}
\titleformat{\subsection}[frame]{\normalfont\large\bfseries}{\stepcounter{epic}ÉPICA\ \theepic}{3pt}{\centering ``#1''}

\titleformat{\subsubsection}[block]{\normalfont\large\bfseries}{\stepcounter{story}\uline{Historia de usuario \thestory : #1}}{0pt}{}

\setcounter{secnumdepth}{3}

\subsection{Como usuario quiero poder jugar a \\ Rolit...}
\subsubsection{... siguiendo un conjunto mínimo de normas}
\textbf{Fase descriptiva}: se establece un conjunto de normas que se deberán cumplir durante el juego.

\textbf{Prioridad}: alta, las normas del juego son la base para crear el mismo.

\textbf{Tamaño estimado}: 1 sprint para la base, aunque se irá mejorando a lo largo del proyecto.

\textbf{Condiciones de aceptación}: el juego es totalmente funcional con el conjunto de normas especificado.

\textbf{Explicación detallada}
Esta historia de usuario fue concebida como tal y ejecutada en el primer sprint. Los \textit{bullet points} a continuación enumerados forman parte de la historia de usuario homónima en el Sprint Backlog 1.
Esta historia de usuario conlleva implementar las siguientes normas en el juego:
\begin{itemize}
\item Consta de un tablero cuadrado de 8x8 casillas
\item Consta de hasta 4 jugadores con colores asignados: amarillo, rojo, verde y azul.
\item Cada usuario puede elegir su nombre como nickname al inicio de la partida.
\item Las piezas del juego son cubos redondos de colores que se introducen en las casillas del tablero.
\item Solo se puede colocar un cubo redondo en una casilla vacía adyacente a una ya ocupada, excepto el primer cubo que se coloca en la posición que se desee.
\item Un cubo se dice atrapado, cuando se encuentra en la línea que une un cubo recién puesto con otro del mismo color.
\item Las direcciones válidas son las líneas rectas dadas por alguna de las casillas adyacentes.
\item Cuando se coloca un cubo, se cambian de color todos los cubos atrapados en las direcciones válidas.
\item En cada turno un jugador pone un único cubo y se pasa turno al siguiente jugador después de haber cambiado de color los posibles cubos atrapados.
\item El juego acaba cuando el tablero se llena completamente.
\item Gana el jugador que más bolas de su color tenga sobre el tablero cuando se termina el juego.
\end{itemize}

\subsubsection{... con una interfaz agradable}
\textbf{Frase descriptiva}: tener una interfaz gráfica que permita al usuario interactuar con el juego de manera sencilla e intuitiva.

\textbf{Prioridad}: media, no es imprescindible para el funcionamiento del juego.

\textbf{Tamaño estimado}: 3 sprints.

\textbf{Condiciones de aceptación}: tener una interfaz gráfica funcional y sin bugs y que pueda ser utilizada por cualquier usuario externo al proyecto.

\textbf{Explicación detallada}
Esta historia de usuario surge fruto de juntar las deprecadas historias de usuario:
\begin{itemize}
\item  \textit{Como usuario, me gustaría que se pudiesen personalizar los colores con los que jugamos cada jugador porque hace más visual el juego. }

    Dicha  ``historia de usuario '', que a la luz de esta refactorización consideramos  ``tarea '' de esta nueva historia de usuario superior, fue implementada en el Sprint 2. Las tareas descritas en el Sprint Backlog 2 pasan a constituir las siguientes subtareas:
	\begin{itemize}
     \item Los jugadores deben poder elegir entre los siguientes colores: rojo, amarillo, azul, rosa, verde, morado, negro, naranja, marrón y beige.
     \end{itemize}

\item  \textit{Como usuario, me gustaría que tenga una interfaz gráfica amable porque hace más fácil jugar.}
\end{itemize}

\subsection{Como usuario quiero que Rolit introduzca características innovadoras...}
\subsubsection{... pensando en las posibilidades que brinda el multijugador}
\textbf{Frase descriptiva}: tener varias modalidades de juego teniendo en cuenta el número de jugadores, así como diferentes tamaños, formas para el tablero, la posibilidad de jugar en red, o con inteligencias artificiales.

\textbf{Prioridad}: media.

\textbf{Tamaño estimado}: 5 sprints.

\textbf{Condiciones de aceptación}: se puede elegir entre varios tamaños y formas para el tablero, y además las modalidades de juego por equipos, jugador automático y clásica son completamente funcionales tanto al jugar con inteligencias artificiales como en red.

\textbf{Explicación detallada}
Esta historia de usuario surge fruto de juntar las deprecadas historias de usuario:
\begin{itemize}
\item  \textit{Como usuario, me gustaría que el número de jugadores fuese variable porque así se adapta a diferentes grupos de personas. }

    Dicha  ``historia de usuario '', que a la luz de esta refactorización consideramos  ``tarea '' de esta nueva historia de usuario superior, fue implementada en el Sprint 2. Las tareas descritas en el Sprint Backlog 2 pasan a constituir las siguientes subtareas:
    \begin{itemize}
  	  \item Cuando el usuario decida crear una nueva partida, se le debe preguntar por el número de jugadores para dicha partida.
      \item El número mínimo de jugadores es 2.
      \item El número máximo de jugadores es 10.
    \end{itemize}
    

\item  \textit{Como usuario, me gustaría que el tamaño del tablero sea variable porque se adapta mejor a los grupos variables. }

    Dicha  ``historia de usuario'', que a la luz de esta refactorización consideramos  ``tarea'' de esta nueva historia de usuario superior, fue implementada en el Sprint 2. Las tareas descritas en el Sprint Backlog 2 pasan a constituir la siguiente subtarea:
    \begin{itemize}
     \item se puede elegir tamaño entre: pequeño (9x9), mediano (13x13), grande (17x17).
	\end{itemize}

\item  \textit{Como usuario, me gustaría que la forma del tablero fuese variable para que cambiasen las estrategias. }
    Ahora hay diferentes formas del tablero, cada una de ellas con los tres tamaños mencionados previamente, estas son:
    \begin{itemize}
    	\item Círculo
    	\item Rombo
    	\item Cuadrado
    \end{itemize}

\item  \textit{Como usuario, me gustaría que se pudiera jugar en red. }
    Existe la posibilidad de jugar tanto en tu propio ordenador como de jugar en red contra otros usuarios.

\item  \textit{Como usuario, me gustaría que se pudiera jugar contra una inteligencia artificial, así como que ellas jugaran solas. }
    Será posible jugar contra inteligencias artificiales, estas cuentan como jugadores, es decir, podrá el número de inteligencias sumado al número de personas debe sobrepasar 2 y no ser más de 10.

\item \textit{Como usuario, me gustaría que hubiese modalidades de juego sobre rolit:}
\begin{itemize}
	 \item  \textit{Clásica} 
     \item  \textit{Por equipos }
\end{itemize}
\end{itemize}


\subsubsection{... siendo intuitivo y cómodo de jugar}
\textbf{Frase descriptiva}: tener un tutorial al inicio del juego, poder cargar y guardar partida, poder ver la repetición de una partida paso a paso y poder salir del juego en cualquier momento durante la ejecución del mismo.

\textbf{Prioridad}: baja, son elementos adicionales a la funcionalidad del juego.

\textbf{Tamaño estimado}: 4 sprints

\textbf{Condiciones de aceptación}: existe la posibilidad de ejecutar un tutorial al inicio del juego para entender las normas, las opciones de cargar y guardar partida se ejecutan correctamente en cualquier momento del juego, y al salir del juego en cualquier momento no se lancen excepciones y funcione correctamente.

\textbf{Explicación detallada}
Esta historia de usuario surge fruto de juntar las deprecadas historias de usuario:
\begin{itemize}
\item  \textit{Como usuario, me gustaría que hubiese un tutorial para entender bien las normas y excepciones del juego.} 

\item  \textit{Como usuario, me gustaría que se pudiera guardar repeticiones de partida para poder revisarlas más tarde.} 

\item  \textit{Como usuario, me gustaría que se pudiese guardar y cargar partida para continuar más tarde porque permite poner en pausa el juego.}
    Ya deprecado antes de esta refactorización, pero fue implementada en el Sprint Backlog 1 bajo las siguientes características:
    \begin{itemize}
    \item Se debe poder guardar la partida y salir del juego en cualquier momento durante la ejecución del mismo.
     \item Al iniciar el juego, debe poderse elegir entre cargar una partida guardada o iniciar una nueva.
     \item Se pueden tener varias partidas guardadas
     \item Para cargar una partida, se muestra una lista para poder elegir la partida a continuar.
     \item Para guardar partida, se debe elegir un nombre identificativo para denotar a dicha partida.
    \end{itemize}
\item  \textit{Como usuario, me gustaría poder guardar y cargar distintas partidas, eligiendo el nombre del fichero donde se cargan/guardan. }
    Se trata de un refinamiento del punto anterior que se introdujo sin dificultades en el Sprint 2.
    
\item  \textit{Como usuario, me gustaría poder salir del juego en cualquier momento (introducida el 17 de febrero de 2022)} 

    Dicha  ``historia de usuario '', que a la luz de esta refactorización consideramos  ``tarea '' de esta nueva historia de usuario superior, fue implementada en el Sprint 2. Las tareas descritas en el Sprint Backlog 2 pasan a constituir las siguientes subtareas:
    \begin{itemize}
    \item La partida debe finalizar cuando cualquier jugador así lo indique.
    \item Se usará un comando  \textit{exit}  para llevar a cabo esta funcionalidad.
    \end{itemize}
\end{itemize}

\defaultformat
\end{document}