\documentclass[../../SCRUM.tex]{subfiles}
\defaultformat

\begin{document}
\subsection{Historias iniciales}
\begin{itemize}
\item
  Como usuario, me gustaría que el número de jugadores fuese variable
  porque así se adapta a diferentes grupos de personas.
\item
  Como usuario, me gustaría que se puediesen personalizar los colores
  con los que jugamos cada jugador porque hace más visual el juego.
\item
  Como usuario, me gustaría que hubiese tableros distintos para poder
  tener distintas estrategias a la hora de jugar.

  \begin{itemize}
  \item
    Como usuario, me gustaría que el tamaño del tablero sea variable
    porque se adapta mejor a los grupos variables.
  \item
    Como usuario, me gustaría que la forma del tablero fuese variable
    para que cambiasen las estrategias.
  \end{itemize}
\item
  Como usuario, me gustaría que hubiese varias modalidades de juego
  sobre rolit:

  \begin{itemize}
  \item
    Clásica
  \item
    Modalidades por tiempo
  \item
    Por equipos
  \item
    Trampa del Ordenador (El ordenador me cambia las cosas)
  \item
    Trampa de cada jugador (Cada jugador tiene un número fijo de
    oportunidades de trampa)
  \item
    Bolas especiales.
  \end{itemize}
\item
  Como usuario, me gustaría que se pudiese guardar y cargar partida para
  continuar más tarde porque permite poner en pausa el juego.
\item
  Como usuario, me gustaría poder poner mi nombre como nickname durante
  la partida.
\item
  Como usuario, quiero que al final se muestre un ranking de puntos de
  la partida.
\item
  Como usuario, me gustaría que tenga una interfaz gráfica amable porque
  hace más fácil jugar.
\item
  Como usuario, me gustaría que se pudiera guardar repeticiones de
  partida para poder revisarlas más tarde.
\item
  Como usuario, me gustaría que hubiese un tutorial para entender bien
  las normas y excepciones del juego.
\item
  Como usuario, me gustaría jugar al rolit siguiendo un conjunto mínimo
  de normas (mirar normas rolit).
\end{itemize}

\subsection{Cambios Sprint 1}
\subsubsection*{Historias añadidas}
\begin{itemize}
\item
  Como usuario, me gustaría poder salir del juego en cualquier momento.
\end{itemize}

\subsection{Cambios Sprint 2}
\subsubsection*{Historias modificadas}
\begin{itemize}
\item
   Como usuario, me gustaría que se pudiese guardar y
  cargar partida para continuar más tarde porque permite poner en pausa
  el juego.{}
\item
  Como usuario, me gustaría poder guardar y cargar distintas partidas,
  eligiendo el nombre del fichero donde se cargan/guardan.
\end{itemize}

\subsection{Cambios Sprint 3}
\subsubsection*{Historias modificadas}

\begin{itemize}
\item
  Como usuario, me gustaría que hubiese varias modalidades de juego
  sobre rolit:

  \begin{itemize}
  \item
    Clásica
  \item
     Modalidades por tiempo{}
  \item
    Por equipos
  \item
     Trampa del Ordenador (El ordenador me cambia las
    cosas)
  \item
     Trampa de cada jugador (Cada jugador tiene un número
    fijo de oportunidades de trampa)
  \item
     Bolas especiales.
  \end{itemize}
\end{itemize}

\subsubsection*{Historias añadidas}
\begin{itemize}
\item
  Como usuario, me gustaría que hubiese varias modalidades de juego
  sobre rolit:

  \begin{itemize}
  \item
    Clásica
  \item
    Por equipos
  \item
    Contra la CPU
  \item
    Online
  \end{itemize}
\end{itemize}

\subsection{Cambios Sprint 4}

\subsubsection*{Historias modificadas}
\begin{itemize}
\item
  Como usuario, me gustaría que hubiese varias modalidades de juego
  sobre rolit:

  \begin{itemize}
  \item
    Clásica
  \item
    Por equipos
  \item
     Contra la CPU
  \item
    \textsubscript{Online}
  \end{itemize}
\end{itemize}

\subsubsection*{Historias añadidas}
\begin{itemize}
\item
  Como usuario quiero que Rolit introduzca características innovadoras
  pensando en las posibilidades que brinda el multijugador:

  \begin{itemize}
  \item
    Como usuario, me gustaría que se pudiera jugar en red.
  \item
    Como usuario, me gustaría que se pudiera jugar contra una
    inteligencia artificial, así como que ellas jugaran solas.
  \end{itemize}
\end{itemize}

\subsection{Cambios Sprint 5}
En el sprint 5 fueron refactorizadas completamente todas las historias
de usuario, manteniendo solo 2 épicas con 2 historias de usuario cada
una. Nos adaptamos al formato de SCRUM, incluyendo frase descriptiva,
prioridades, tamaños, condiciones de aceptación y explicación detallada.
El resultado puede verse en:

\end{document}
