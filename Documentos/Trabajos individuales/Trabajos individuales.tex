\documentclass[../FINAL/Scrum/SCRUM.tex]{subfiles}
\defaultformat

\begin{document}
\section{Sergio Miguel García Jiménez}
\subsection*{Código}

\begin{itemize}
\item Crear parte del menú principal en modo consola.
\item Colaborar en el diseño de la clase Player.
\item Desarrollar gran parte de la lógica de cargar, guardar y borrar partida (SaveLoadManager).
\item Contribuir a que modelo trabaje exclusivamente con JSON. Esto incluía la concepción del diseño de reports así como su implementación en las distintas clases, tarea inicialmente desarrollada por Leo, Virginia y yo, y continuada por el resto de compañeros.
\item Desarrollé junto con Leo toda la funcionalidad de red, incluyendo elementos de la GUI que precisa, su extensión a todos los modos de juego y la lógica detrás de las conexiones.
\item Me encargué junto a otros del desarrollo de la estructura de la GUI; concretamente en lo referente al tablero, diálogos de cargar, crear, borrar partida y la interacción de la GUI con el modelo, así como una participación activa en su pulido y debug por las posteriores características introducidas por el modelo.
\item Realicé diversas labores de optimización para solventar un pobre rendimiento de la GUI a la hora de mostrar el tablero, actualizándose ahora en tiempo casi instantáneo y constante.
\item El debug de las características que estaba desarrollando conllevó a debugear otras clases referentes al modelo para perfeccionarlo.
\item Encargado de realizar algunos merge y etiquetado.
\item Proponer y solventar algunas de las issues.
\end{itemize}

\subsection*{Documentación}
\begin{itemize}
\item He realizado los primeros diagramas de clases y UML para ofrecer unos primeros prototipos para poder determinar por qué estilo de diseño queríamos adoptar.
\item Concebí el diseño de la refactorización de las historias de usuario para adaptarlas al formato de la asignatura, tarea después completada por compañeros.
\item Creación de los primeros artículos y la estructura preliminar de la Wiki en los primeros sprints.
\item Trabajo UML preliminar con Modelio.
\item Pugnar por pasar de trabajar en el software Modelio a PlantUML para trabajar con mayor comodidad.
\end{itemize}

\titleformat{\section}{\normalfont\Large\bfseries}{}{0pt}{\newpage #1}
\section{Leonardo Macías Sánchez}
\subsection*{Código}
\begin{itemize}
\item Concepción, diseño y desarrollo de la clase SaveLoadManager, que en sprints futuros sería refactorizada por Sergio para adaptarla a las nuevas necesidades.

\item Desarrollo de las clases Player, Cube y Color, sobre todo durante la etapa inicial del proyecto.

\item Refactorización de la clase Board para que contuviera la lógica del juego.

\item Desarrollo y adaptación del juego a la red, incluyendo una extensión de la interfaz gráfica que facilitase el funcionamiento al usuario. Esta tarea fue desarrollada conjuntamente con Sergio.

\item Diseño y desarrollo del paquete replay, que abarca la interfaz Replayable y las clases Replay y GameState. Además de implementar toda la lógica, también me he dedicado a su correcta visualización, tanto en consola como en GUI.

\item Principal impulsor y defensor del uso de JSONObjects en el proyecto. Ideé la interfaz Reportable y me encargué del diseño de los JSON y de su posterior implementación junto con Sergio y Virginia.

\item Diseño e implementación de los métodos toString() del modelo.

\item Colaboración en el diseño de los GameBuilders junto con Juan Diego y Daniel.

\item Refactorización, junto a Juan Diego, de la clase Game para poder introducir las modalidades de GameTeams y GameClassic.

\item Ayudar a Virginia y Mar con problemas que surgieron al crear JUnits.

\item Creación del ranking junto con Mar.

\item Refactorización completa de todas las clases de la GUI, rediseñandola al completo (a excepción del tablero) y creando nuestros propios componentes visuales, los RolitComponents.

\item Propuse la creación de la clase TurnManager.

\item Labores de debug.

\end{itemize}

\subsection*{Documentación}
\begin{itemize}
\item Diseño de los reports.
\item Redactar el funcionamiento de los Builders.
\item Desarrollo de diagramas de secuencia.
\item Redactar algunos de los sprint reviews y retrospectives.
\item Colaboración en el desarrollo de la wiki.
\item Búsqueda de una alternativa a Modelio para el desarrollo de diagramas de clases.
\end{itemize}

\section{Daniel González Arbelo}
\subsection*{Código}
\begin{itemize}
\item Desarrollo en las etapas más tempranas del proyecto de las clases Game y Board, implementando los métodos encargados de la ejecución y actualización del juego en los turnos.
\item Retoques y debug en las primeras etapas de la clase SaveLoadManager para el funcionamiento de cargado y guardado de partida.
\item Creación de las clases herederas de Command, junto con Mar, aplicándose el patrón comando.
\item Desarrollo de la lectura de datos de entrada para la creación del juego por consola junto con Leo y Mar.
\item Creación de los ficheros para las formas de los tableros y las imágenes de las celdas en GUI.
\item Creación de las primeras versiones funcionales de la GUI  (ventanas de creación de juego, tablero, ranking, etc.), desarrolladas todas conjuntamente con Sergio.
\item Creación, desarrollo y debug de las inteligencias artificiales y sus estrategias (y desarrollo de algunas alternativas para su funcionamiento con Juan Diego, que fueron finalmente deprecadas).
\item Resolución de algún fallo en el funcionamiento de la red.
\item Refactorización, junto con Juan Diego, del modelo para que trabajase en su propia hebra, y de la vista por consola.
\item Refactorización del manejo de turnos (con la creación de TurnManager, idea elaborada con Leo) y de la clase Player para adaptarlos al funcionamiento con la hebra del modelo y las inteligencias artificiales.
\item Creación de la lógica encargada del funcionamiento del tutorial.
\item Tareas de debug, manejo de excepciones y resolución de issues.
\end{itemize}

\subsection*{Documentación}
\begin{itemize}
\item Creación de los primeros diagramas de secuencia para fijar el formato.
\item Establecimiento de un procedimiento para el desarrollo de los diagramas de secuencia usando PlantUML y IntelliJ.
\item Documentación acerca del funcionamiento de la GUI.
\item Participación en los sprint reviews y retrospectives.

\end{itemize}


\section{María del Mar Ramiro Ortega}
\subsection*{Código}
\begin{itemize}
\item Desarrollo de un primer modelo de ranking, mostrado al final de la partida y ordenado por puntos.
\item Creación de las clases herederas de Command, junto con Daniel, aplicándose el patrón comando.
\item Diseño, junto con Daniel, y desarrollo de los distintos tamaños y formas del tablero, tanto a nivel de lógica del juego como a nivel de su carga y guardado por medio de la clase SaveLoadManager.
\item Realización de una versión incial de los JUnits con mi compañera Virginia.
\item Actualización de los JUnits en todas las refactorizaciones siguientes junto con Sergio.
\item Añadido de nuevos tests a los JUnits de las distintas historias de usuario que íbamos implementando.
\item Creación del ranking de GUI junto con Leo y mejora del ranking de console.
\end{itemize}

\subsection*{Documentación}
\begin{itemize}
\item Creación de diagramas de secuencia.
\item Desarrollo de una refactorización de las Historias de Usuario junto con Virginia, que fueron actualizadas y subidas a la wiki.
\item Documento acerca de los diversos cambios que hicimos durante los distintos sprints junto con Leo, así como su actualización en la wiki.
\item Redactar algún sprint review y retrospective, así como la actualización de la wiki respecto a varios sprints reviews y retrospectives que realizamos en PDF.
\item Participación en los sprint reviews y retrospectives, junto con el resto de mis compañeros.
\item Javadoc del paquete Builder y de la vista de consola.
\item Documentación y UML sobre la HU \textit{Como usuario quiero poder jugar a Rolit siguiendo un conjunto mínimo de normas }.
\end{itemize}

\section{Juan Diego Barrado Daganzo}
\subsection*{Código}
\begin{itemize}

\item Desarrollo de la lógica interna de Game en las fases iniciales del proyecto, creación de la clase Player e integración de la misma sobre el modelo.

\item Desarrollo del mecanismo de cambio de puntuaciones a través de los Cube, que permitía encapsular mucho mejor la puntuación de cada jugador.

\item Diseño de las responsabilidades conceptuales de los integrantes del modelo, es decir, asignar las capacidades y dependencias del Board, Game, Player... tanto a nivel de código como a nivel de concepto, lo que unificó lo que se esperaba de cada clase e hizo más fácil la integración de funcionalidades posteriores.

\item Creación y diseño de la clase builder y de la refactorización del juego a través de factorías.

\item Encapsulación del modelo en la clase abstracta de Game, definición de las responsabilidades de cada una de las clases hijas (GameClassic y GameTeams) e implementación de las mismas.

\item Adaptación del Modelo-Vista-Controlador a través de la generación de un hilo propio para el modelo que permitía una mejor integración de los jugadores automáticos. Creación de la cola de prioridad de Cube para poder utilizar el modo red y los jugadores automáticos de forma simultánea.

\item Creación de la vista de consola y generación de la interfaz \textit{ConsoleWindow} para mantener un criterio uniforme a modo de ``componentes'' tal y como se hace en Swing, lo que simplificó la independencia del modelo y la vista por su similitud con las ventanas de Swing.

\item Depurado de la interfaz de RolitObserver y de los métodos de notificación a observadores.

\item Depurado de la red para integrar los jugadores automáticos.
\end{itemize}
\subsection*{Documentación}
\begin{itemize}
\item Rediseñado de los reports iniciales.
\item Creación del formato de los documentos de desarrollo (los utilizados durante los sprint por el equipo de desarrollo, no los que se incluyen en la entrega) para informar a todos los participantes de la implementación final que se dió al diseño planeado de forma general.
\end{itemize}

\section{Virginia Chacón Pérez}
\subsection*{Código}
\begin{itemize}
\item Diseño, desarrollo e implementación de los JSONObjects y de la interfaz Reportable junto con Leo y Sergio.
\item Diseño y desarrollo de los tests de JUnit junto con Mar.
\item Ayuda con la refactorización de la GUI a Leo.
\end{itemize}
\subsection*{Documentación}
\begin{itemize}
\item Primera en empezar a entender y a trabajar con Modelio, aunque luego descubriésemos otros programas más útiles para nuestro proyecto en concreto.
\item Diseño de los primeros diagramas de clases con Modelio.
\item Diseño de algunos diagramas de secuencia, tanto en IntelliJ como en PlantUML.
\item Creación de la parte de Historias de Usuario en la Wiki y actualización de la misma con Mar.
\item Colaboración en otras partes de la Wiki.
\item Participación en los Sprint Reviews y Retrospectives junto con el resto de mis compañeros.
\item Creación y edición del vídeo tutorial para entender las normas del juego.
\end{itemize}

\end{document}