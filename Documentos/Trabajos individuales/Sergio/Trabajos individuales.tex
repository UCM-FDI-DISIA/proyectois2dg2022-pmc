\documentclass{article}
\begin{document}
\title{Documento trabajo individual}
\maketitle
\section*{Sergio Miguel García Jiménez}
\subsection*{Sprint 1}
\subsubsection*{Código}

\begin{itemize}
\item Crear parte del menú principal en modo consola.
\item Colaborar en el diseño de la clase Player.
\item Desarrollar gran parte de la lógica de cargar, guardar y borrar partida (SaveLoadManager).
\item Contribuir a que modelo trabaje exclusivamente con JSON. Esto incluía la concepción del diseño de reports así como su implementación en las distintas clases, tarea inicialmente desarrollada por Leo, Virginia y yo, y continuada por el resto de compañeros.
\item Desarrollé junto con Leo toda la funcionalidad de red, incluyendo elementos de la GUI que precisa, su extensión a todos los modos de juego y la lógica detrás de las conexiones.
\item Me encargué junto a otros del desarrollo de la estructura de la GUI; concretamente en lo referente al tablero, diálogos de cargar, crear, borrar partida y la interacción de la GUI con el modelo, así como una participación activa en su pulido y debug por las posteriores características introducidas por el modelo.
\item Realicé diversas labores de optimización para solventar un pobre rendimiento de la GUI a la hora de mostrar el tablero, actualizándose ahora en tiempo casi instantáneo y constante.
\item El debug de las características que estaba desarrollando conllevó a debugear otras clases referentes al modelo para perfeccionarlo.
\item Encargado de realizar algunos merge y etiquetado.
\item Proponer y solventar algunas de las issues.
\end{itemize}

\subsubsection*{Documentación}
\begin{itemize}
\item He realizado los primeros diagramas de clases y UML para ofrecer unos primeros prototipos para poder determinar por qué estilo de diseño queríamos adoptar.
\item Concebí el diseño de la refactorización de las historias de usuario para adaptarlas al formato de la asignatura, tarea después completada por compañeros.
\item Creación de los primeros artículos y la estructura preliminar de la Wiki en los primeros sprints.
\item He realizado los primeros diagramas de clases y UML para ofrecer unos primeros prototipos para poder determinar por qué estilo de diseño queríamos adoptar.
\item Trabajo UML preliminar con Modelio.
\item Pugnar por pasar de trabajar en el software Modelio a PlantUML para trabajar con mayor comodidad.
\end{itemize}

\subsection*{Sprint 2}
\subsection*{Sprint 3}
\subsection*{Sprint 4}
\subsection*{Sprint 5}
\subsection*{Sprint 6}
\subsection*{Sprint 7}
\end{document}