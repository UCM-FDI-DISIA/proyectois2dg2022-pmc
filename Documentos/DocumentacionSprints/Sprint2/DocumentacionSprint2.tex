\documentclass[../../FINAL/Scrum/SCRUM.tex]{subfiles}
\defaultformat

\begin{document}
\section{SPRINT 2}
\subsection{Revisión del Sprint}
Respecto a los objetivos esperados del Sprint Backlog, se ha implementado adecuadamente la funcionalidad de ``\textit{Como usuario, me gustaría poder guardar y cargar distintas partidas.}''. El usuario es capaz de guardar un instante de una partida en cualquier momento y darle un nombre. Para cargar la partida, hay que introducir el nombre del archivo al seleccionar ``cargar partida``.

En cuanto a ``\textit{Como usuario, me gustaría poder salir del juego en cualquier momento.}'', se ha introducido al juego un comando exit que permite llevar a cabo esta funcionalidad.

El usuario puede elegir con libertad los nicknames que prefiera para jugar a rolit y su color, así como el número de jugadores, luego las funcionalidades ``\textit{Como usuario, me gustaría que el número de jugadores fuese variable para adaptarse mejor a diferentes grupos de personas.}'' y ``\textit{Como usuario, me gustaría que se pudiesen personalizar los colores con los que jugamos cada uno para hacer más visual el juego.}''  están implementadas satisfactoriamente.

Además, al comenzar una partida nueva se le da al usuario la opción de elegir el tamaño del tablero, por lo que la historia de usuario ``\textit{Como usuario, me gustaría que hubiese distintos tamaños de tablero seleccionables.}'' también ha sido ejecutada.

Paralelamente a la implementación de nuevas historias de usuario, hemos refactorizado gran parte del código para facilitar su desarrollo y manejo de cara a futuros sprints. Los cambios se pueden consultar en el documento \textit{RefactorizacionSprint1} dentro de la carpeta \textit{DocumentosDesarrollo}.

\subsection{Retrospectiva}
\subsubsection*{More Of}
\begin{itemize}
\item Es necesario que los miembros del equipo adquieran más experiencia con el manejo de Git y GitHub, así como los flujos de trabajo para Sistemas de Control de Versiones Descentralizados. Se observa una mejora notable con respecto al sprint anterior.

\item Reservar más tiempo para la depuración del código.

\item Mejorar el flujo de trabajo definido para que se adapte bien al formato distribuido de Git y que permita trabajar de forma paralela entre miembros con distintas funcionalidades.

\end{itemize}
\subsubsection*{Keep Doing}
\begin{itemize}
\item Trabajar de forma paralela y asignar una tarea (no siempre la misma, para tener una visión global del proyecto) a grupos de entre los miembros del equipo ha agilizado la producción de software.

\item Mantener a todos los miembros al tanto de los cambios, modificaciones o direcciones que toma el proyecto a lo largo del proyecto.

\item Mantener el grado de comunicación entre los miembros del equipo y la periodicidad de la misma.

\item Continuar con las reuniones dos días a la semana en los que reservamos un aula de trabajo en grupo para realizar avances del proyecto de forma paralela a los días de clase.

\end{itemize}



\subsubsection*{Start Doing}
\begin{itemize}
\item Fijar con antelación una reunión para llevar a cabo el Sprint Review.

\end{itemize}

\subsubsection*{Stop Doing}


\subsubsection*{Less of}
\begin{itemize}
\item Deberíamos dedicarle algo menos de tiempo al código en favor de la documentación.
\end{itemize}

\subsection{Planificación del siguiente Sprint}
Durante el siguiente Sprint, el equipo de desarrollo llevará a cabo las siguientes historias de usuario del Product Backlog:
\begin{itemize}
\item \textit{Como usuario, me gustaría que se pudiera guardar repeticiones de partida para poder revisarlas más tarde.}
\item \textit{Como usuario, me gustaría que se pudiese jugar a la versión por equipos de Rolit.}
\item \textit{Como usuario, me gustaría que se hubiese distintas formas de tableros seleccionables.}
\end{itemize}
Además, durante este Sprint el trabajo se va a distribuir de la siguiente manera:
\begin{itemize}
\item En primer lugar, se discutirá la necesidad de una refactorización de Controller, y se ejecutará en caso de que así se acuerde.
\item Se solventarán algunas deudas de código existentes de la versión anterior.
\item El equipo se dedicará generar el código para las nuevas funcionalidades. En el tiempo restante se tratará de mejorar algunas características del sprint anterior.
\end{itemize}
\defaultformat
\end{document}
