\documentclass[../../SCRUM.tex]{subfiles}
\defaultformat

\begin{document}
\section{SPRINT 6}
\subsection{Revisión del Sprint}
La funcionalidad de ``\textit{Como usuario, me gustaría que se pudiera jugar contra una inteligencia artificial, así como que ellas jugaran solas}'' ha sido completada con éxito. Existen 3 niveles de dificultad seleccionables para IA que pueden jugar entre ellas o contra otros jugadores físicos.

La característica ``\textit{Como usuario, me gustaría que hubiese un tutorial para entender bien las normas y excepciones del juego}'' ha sido pospuesta por retrasos en la depuración y funcionamiento correcto del juego general, lo que imposibilitaba que se pudiese hacer una simulación a modo de tutorial.

La Historia de Usuario ``\textit{Como usuario, me gustaría poder jugar a Rolit con una interfaz agradable}'' ha sido por fin completada en su totalidad al adaptar la consola para seguir jugando a Rolit, además se ha hecho un lavado de cara a la GUI que hace más intuitivo y agradable el juego.

La funcionalidad ``\textit{Como usuario, me gustaría que se pudiera jugar en red}'' ha sido ampliada para que se puedan jugar los distintos modos de juego también en red. Además, se han introducido nuevas ventanas en la vista para aumentar el feedback que recibe el usuario cuando se conecta a un servidor, mejorando también la Historia de Usuario ``\textit{Como usuario, me gustaría poder jugar a Rolit con una interfaz agradable}''.

\subsection{Retrospectiva}
\subsubsection*{More Of}
\begin{itemize}
\item Es necesario poner al día los diagramas UML del proyecto.
\item Es necesario poner al día los documentos de desarrollo que se han venido haciendo.
\item Deberíamos ir dejando cerrados los \textit{issues} marcados en GitHub.
\end{itemize}

\subsubsection*{Keep Doing}
\begin{itemize}
\item Trabajar de forma paralela y asignar una tarea (no siempre la misma, para tener una visión global del proyecto) a grupos de entre los miembros del equipo ha agilizado la producción de software.

\item Mantener a todos los miembros al tanto de los cambios, modificaciones o direcciones que toma el proyecto a lo largo del proyecto.

\item Mantener el grado de comunicación entre los miembros del equipo y la periodicidad de la misma.

\item Continuar con las reuniones dos días a la semana en los que reservamos un aula de trabajo en grupo para realizar avances del proyecto de forma paralela a los días de clase.

\item Seguir asignando suficiente tiempo para la depuración del código.

\item Fijar con antelación una reunión para llevar a cabo el Sprint Review.

\item Seguir haciendo reuniones de debate sobre cuál es el mejor diseño antes de empezar a programar.
\end{itemize}

\subsubsection*{Start Doing}
\begin{itemize}
\item Hacer las planificaciones de los sprint un poco más exhaustivas, dejando bien marcadas las responsabilidades y las necesidades a priori de cada Historia de Usuario.
\end{itemize}

\subsubsection*{Stop Doing}


\subsubsection*{Less of}
\begin{itemize}
\item Deberíamos dedicarle algo menos de tiempo al código en favor de la documentación.
\end{itemize}

\subsection{Planificación del siguiente Sprint}
Durante el siguiente Sprint, el equipo de desarrollo llevará a cabo las siguientes tareas de la historia de usuario ``\textit{Como usuario quiero que Rolit introduzca características innovadoras siendo intuitivo y cómodo de jugar}'':
\begin{itemize}
\item \textit{Como usuario, me gustaría que hubiese un tutorial para entender bien las normas y excepciones del juego.}
\item Nos vamos a dedicar a solucionar posibles bugs y poner tensión a las situaciones de juego para comprobar que las soluciones son consistentes.
\end{itemize}

\end{document}