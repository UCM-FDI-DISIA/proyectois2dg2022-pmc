\documentclass[../../FINAL/Scrum/SCRUM.tex]{subfiles}
\defaultformat

\begin{document}
\section{SPRINT 5}
\subsection{Revisión del Sprint}
En primer lugar, la funcionalidad de \textit{``Como usuario, me gustaría
que se pudiera jugar en red.''} ha sido completada con éxito para el
modo GameClassic a falta de determinar si debería implementarse para
GameTeams.

La funcionalidad de \textit{``Como usuario, me gustaría que se pudiera
jugar contra una inteligencia artificial, así como que ellas jugaran
solas.''} ha sido completada en gran medida en su fase Beta. No
obstante, precisa refactorización y por ello hemos optado por no
incluirla en el código principal en este sprint, ya que dificultaba la
funcionalidad del juego en red.

\subsection{Retrospectiva}

\subsubsection*{More Of}
\begin{itemize}
\item
  Mantener a todos los miembros al tanto de los cambios, modificaciones
  o direcciones que toma el proyecto a lo largo del proyecto.
\item
  Es necesario poner al día los diagramas UML del proyecto.
\end{itemize}

\subsubsection*{Keep Doing}
\begin{itemize}
\item
  Seguir manejando de forma planificada y eficiente el uso de git y
  gitHub, aprovechándonos de por ejemplo, la posibilidad de crear ramas.
\item
  Trabajar de forma paralela y asignar una tarea (no siempre la misma
  para tener una visión global del proyecto) a grupos de entre los
  miembros del equipo ha agilizado la producción de software.
\item
  Mantener el grado de comunicación entre los miembros del equipo y la
  periodicidad de la misma.
\item
  Continuar con las reuniones dos días a la semana en los que reservamos
  un aula de trabajo en grupo para realizar avances del proyecto de
  forma paralela a los días de clase.
\item
  Seguir asignando suficiente tiempo para la depuración del código.
\item
  Fijar con antelación una reunión para llevar a cabo el Sprint Review.
\item
  Seguir haciendo reuniones de debate sobre cuál es el mejor diseño
  antes de empezar a programar.
\item
  Seguir haciendo la documentación con un tiempo prudencial antes del
  final del sprint.
\end{itemize}

\subsubsection*{Start Doing}

\subsubsection*{Stop Doing}

\subsubsection*{Less of}
\begin{itemize}
\item
  Deberíamos dedicarle algo menos de tiempo al código en favor de la
  documentación.
\end{itemize}

\subsection{Planificación del siguiente Sprint}
Durante el siguiente Sprint, el equipo de desarrollo concluirá las siguientes tareas de la historia de usuario ``\textit{Como usuario quiero que Rolit introduzca características innovadoras pensando en las posibilidades que brinda el multijugador}'' del Product Backlog:
\begin{itemize}
\item ``\textit{Como usuario, me gustaría que se pudiera jugar
contra una inteligencia artificial, así como que ellas jugaran solas.}''
\end{itemize}

Además también realizaremos las tareas de la historia de usuario ``\textit{Como usuario, me gustaría que Rolit fuera intuitivo y cómodo de jugar}'' del Product Backlog:
\begin{itemize}
\item  ``\textit{Como usuario, me gustaría que hubiese un tutorial para entender bien las normas y excepciones del
juego.}''
\end{itemize}

De cara al código deberíamos terminar todos los TODO, FIXME e ``issues'', así como la refactorización del mismo y el tratamiento de las excepciones.

Deberíamos continuar actualizando y mejorando la documentación. Siendo de vital importancia los UML de sprints anteriores y del actual.
\defaultformat
\end{document}
