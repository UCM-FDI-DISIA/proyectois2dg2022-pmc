% Options for packages loaded elsewhere
\PassOptionsToPackage{unicode}{hyperref}
\PassOptionsToPackage{hyphens}{url}
%
\documentclass[
]{article}
\usepackage{amsmath,amssymb}
\usepackage{lmodern}
\usepackage{iftex}
\ifPDFTeX
  \usepackage[T1]{fontenc}
  \usepackage[utf8]{inputenc}
  \usepackage{textcomp} % provide euro and other symbols
\else % if luatex or xetex
  \usepackage{unicode-math}
  \defaultfontfeatures{Scale=MatchLowercase}
  \defaultfontfeatures[\rmfamily]{Ligatures=TeX,Scale=1}
\fi
% Use upquote if available, for straight quotes in verbatim environments
\IfFileExists{upquote.sty}{\usepackage{upquote}}{}
\IfFileExists{microtype.sty}{% use microtype if available
  \usepackage[]{microtype}
  \UseMicrotypeSet[protrusion]{basicmath} % disable protrusion for tt fonts
}{}
\makeatletter
\@ifundefined{KOMAClassName}{% if non-KOMA class
  \IfFileExists{parskip.sty}{%
    \usepackage{parskip}
  }{% else
    \setlength{\parindent}{0pt}
    \setlength{\parskip}{6pt plus 2pt minus 1pt}}
}{% if KOMA class
  \KOMAoptions{parskip=half}}
\makeatother
\usepackage{xcolor}
\IfFileExists{xurl.sty}{\usepackage{xurl}}{} % add URL line breaks if available
\IfFileExists{bookmark.sty}{\usepackage{bookmark}}{\usepackage{hyperref}}
\hypersetup{
  hidelinks,
  pdfcreator={LaTeX via pandoc}}
\urlstyle{same} % disable monospaced font for URLs
\setlength{\emergencystretch}{3em} % prevent overfull lines
\providecommand{\tightlist}{%
  \setlength{\itemsep}{0pt}\setlength{\parskip}{0pt}}
\setcounter{secnumdepth}{-\maxdimen} % remove section numbering
\ifLuaTeX
  \usepackage{selnolig}  % disable illegal ligatures
\fi

\author{}
\date{}

\begin{document}

\begin{center}\rule{0.5\linewidth}{0.5pt}\end{center}

\hypertarget{sprint-5}{%
\section{SPRINT 5}\label{sprint-5}}

17 de abril de 2022 ***

\hypertarget{revisiuxf3n-del-sprint}{%
\subsection{REVISIÓN DEL SPRINT}\label{revisiuxf3n-del-sprint}}

En primer lugar, la funcionalidad de \emph{``Como usuario, me gustaría
que se pudiera jugar en red.''} ha sido completada con éxito para el
modo GameClassic a falta de determinar si debería implementarse para
GameTeams.

La funcionalidad de \emph{``Como usuario, me gustaría que se pudiera
jugar contra una inteligencia artificial, así como que ellas jugaran
solas.''} ha sido completada en gran medida en su fase Beta. No
obstante, precisa refactorización y por ello hemos optado por no
incluirla en el código principal en este sprint, ya que dificultaba la
funcionalidad del juego en red.

\hypertarget{retrospectiva}{%
\subsection{RETROSPECTIVA}\label{retrospectiva}}

\hypertarget{more-of}{%
\subsubsection{More Of}\label{more-of}}

\begin{itemize}
\item
  Mantener a todos los miembros al tanto de los cambios, modificaciones
  o direcciones que toma el proyecto a lo largo del proyecto.
\item
  Es necesario poner al día los diagramas UML del proyecto.
\end{itemize}

\hypertarget{keep-doing}{%
\subsubsection{Keep Doing}\label{keep-doing}}

\begin{itemize}
\item
  Seguir manejando de forma planificada y eficiente el uso de git y
  gitHub, aprovechándonos de por ejemplo, la posibilidad de crear ramas.
\item
  Trabajar de forma paralela y asignar una tarea (no siempre la misma
  para tener una visión global del proyecto) a grupos de entre los
  miembros del equipo ha agilizado la producción de software.
\item
  Mantener el grado de comunicación entre los miembros del equipo y la
  periodicidad de la misma.
\item
  Continuar con las reuniones dos días a la semana en los que reservamos
  un aula de trabajo en grupo para realizar avances del proyecto de
  forma paralela a los días de clase.
\item
  Seguir asignando suficiente tiempo para la depuración del código.
\item
  Fijar con antelación una reunión para llevar a cabo el Sprint Review.
\item
  Seguir haciendo reuniones de debate sobre cuál es el mejor diseño
  antes de empezar a programar.
\item
  Seguir haciendo la documentación con un tiempo prudencial antes del
  final del sprint.
\end{itemize}

\hypertarget{start-doing}{%
\subsubsection{Start Doing}\label{start-doing}}

\hypertarget{stop-doing}{%
\subsubsection{Stop Doing}\label{stop-doing}}

\hypertarget{less-of}{%
\subsubsection{Less of}\label{less-of}}

\begin{itemize}
\tightlist
\item
  Deberíamos dedicarle algo menos de tiempo al código en favor de la
  documentación.
\end{itemize}

\hypertarget{planificaciuxf3n-del-siguiente-sprint}{%
\subsection{PLANIFICACIÓN DEL SIGUIENTE
SPRINT}\label{planificaciuxf3n-del-siguiente-sprint}}

Durante el siguiente Sprint, el equipo de desarrollo concluirá la
siguiente tarea \emph{Como usuario, me gustaría que se pudiera jugar
contra una inteligencia artificial, así como que ellas jugaran solas.}
de la historia de usuario \emph{``Como usuario quiero que Rolit
introduzca características innovadoras pensando en las posibilidades que
brinda el multijugador''} del Product Backlog.

Además también realizaremos la tarea \emph{Como usuario, me gustaría que
hubiese un tutorial para entender bien las normas y excepciones del
juego.} de la historia de usuario \emph{``Como usuario, me gustaría que
Rolit fuera intuitivo y cómodo de jugar''} del Product Backlog.

De cara al código deberíamos terminar todos los TODO, FIXME e
``issues'', así como la refactorización del mismo y el tratamiento de
las excepciones.

Deberíamos continuar actualizando y mejorando la documentación. Siendo
de vital importancia los UML de sprints anteriores y del actual.

\end{document}
