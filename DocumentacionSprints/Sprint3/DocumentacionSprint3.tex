\documentclass{article}
\title{SPRINT 3}
\date{\today}
\author{Grupo PMC}
\usepackage[spanish]{babel}
\usepackage[top=2.5cm, bottom=2.5cm, left=3cm, right=3cm]{geometry}
\clubpenalty=10000 %líneas viudas NO
\widowpenalty=10000 %líneas viudas NO

\begin{document}
\maketitle
\section{REVISIÓN DEL SPRINT}
En primer lugar, la funcionalidad "\textit{Como usuario, me gustaría que se pudiera guardar repeticiones de partida para poder revisarlas más tarde.}" ha sido implementada exitosamente. Resulta cómodo revisar repeticiones, tal y como se deseaba.

Aparte, el cargado de partidas al inicio de la ejecución es ahora más cómodo y amigable con el usuario. Al seleccionar la opción de cargar partida se muestra una lista con las partidas disponibles para ser cargadas, de forma que el usuario no tiene que conocer el nombre de los ficheros en los que han sido guardadas previamente.

Hemos podido dedicar también una buena parte del esfuerzo de este Sprint a la refactorización de código. Esto ha sido muy útil y lo seguirá siendo a medida que sigamos avanzando con el proyecto, pues nos ha facilitado en gran medida el desarrollo del código, lo ha hecho más entendible y hemos conseguido una mayor modularidad, de manera que implementar cambios ahora es mucho más sencillo.

Por otro lado, la funcionalidad "\textit{Como usuario, me gustaría que hubiese distintas formas de tableros seleccionables.}" también se ha logrado como esperábamos. Ahora hay tres formas para elegir: cuadrado, círculo y rombo. Aparte, para cada una de estas formas hay tres tamaños de tableros: pequeño, mediano y grande.

Finalmente, hay una funcionalidad que no hemos podido implementar en este Sprint, que es la funcionalidad de "\textit{Como usuario, me  gustaría que se pudiese jugar a la versión por equipos de Rolit.}". Una causa de esto ha sido que hemos dedicado bastante tiempo de este Sprint a desarrollar UML, por lo que, a pesar de que habría sido posible implementar esta funcionalidad, habría sido a coste de hacerlo con un código menos modular, perdiéndose en parte lo logrado con la refactorización. Por tanto, hemos considerado más conveniente dejar esta funcionalidad de cara al Sprint siguiente, puesto que lo podremos hacer mejor.

\section{RETROSPECTIVA}
\subsection{More Of}
\begin{itemize}
\item Es necesario que los miembros del equipo adquieran más experiencia con el manejo de Git y GitHub, así como los flujos de trabajo para Sistemas de Control de Versiones Descentralizados. Se observa una mejora notable con respecto al sprint anterior.

\item Mejorar el flujo de trabajo definido para que se adapte bien al formato distribuido de Git y que permita trabajar de forma paralela entre miembros con distintas funcionalidades.

\end{itemize}
\subsection{Keep Doing}
\begin{itemize}
\item Trabajar de forma paralela y asignar una tarea (no siempre la misma, para tener una visión global del proyecto) a grupos de entre los miembros del equipo ha agilizado la producción de software.

\item Mantener a todos los miembros al tanto de los cambios, modificaciones o direcciones que toma el proyecto a lo largo del proyecto.

\item Mantener el grado de comunicación entre los miembros del equipo y la periodicidad de la misma.

\item Continuar con las reuniones dos días a la semana en los que reservamos un aula de trabajo en grupo para realizar avances del proyecto de forma paralela a los días de clase.

\item Seguir asignando suficiente tiempo para la depuración del código.

\end{itemize}



\subsection{Start Doing}
\begin{itemize}
\item Fijar con antelación una reunión para llevar a cabo el Sprint Review.

\item Mantener más reuniones a lo largo de la semana, independientemente del número de integrantes que puedan asistir.

\end{itemize}

\subsection{Stop Doing}


\subsection{Less of}
\begin{itemize}
\item Deberíamos dedicarle algo menos de tiempo al código en favor de la documentación.
\end{itemize}

\section{PLANIFICACIÓN DEL SIGUIENTE SPRINT}
Durante el siguiente Sprint, el equipo de desarrollo llevará a cabo las siguientes historias de usuario del Product Backlog:
\begin{itemize}
\item \textit{Como usuario, me gustaría que se pudiese jugar a la versión por equipos de Rolit.}.
\item \textit{Como usuario, quiero poder jugar a Rolit con una interfaz agradable.}

\end{itemize}
Además, durante este Sprint el trabajo se va a distribuir de la siguiente manera:
\begin{itemize}
\item En primer lugar, se crearán grupos a los que asignarles estas historias de usuario, y otro grupo para desarrollar los tests de JUnit.
\item Se trabajará en saldar la deuda de documentación pendiente desde los Sprints anteriores.

\end{itemize}

\end{document}