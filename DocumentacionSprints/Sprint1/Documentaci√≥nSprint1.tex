\documentclass{article}
\title{SPRINT 1}
\date{\today}
\author{Grupo PMC}
\usepackage[spanish]{babel}
\usepackage[top=2.5cm, bottom=2.5cm, left=3cm, right=3cm]{geometry}
\clubpenalty=10000 %líneas viudas NO
\widowpenalty=10000 %líneas viudas NO

\begin{document}
\maketitle
\section{REVISIÓN DEL SPRINT}
Respecto a los objetivos esperados del Sprint Backlog, se ha implementado adecuadamente la funcionalidad de ``\textit{Como usuario, me gustaría jugar al rolit siguiendo un conjunto mínimo de normas}''. El usuario es capaz de desarrollar con normalidad una partida del juego rolit sin bugs  conocidos.

En cuanto a ``\textit{Como usuario, me gustaría poder guardar y cargar partida.}'', la funcionalidad se ha implementado correctamente, aunque el método es rudimentario y está obligado a futuras modificaciones para aportarle robustez.

El usuario puede elegir con libertad los nicknames que prefiera para jugar a rolit, luego la funcionalidad ``\textit{Como usuario, me gustaría poder poner mi nombre como nickname durante la partida}'' está implementada satisfactoriamente.

Al finalizar la partida, se muestra un ranking que permite ver los jugadores y sus puntuaciones, lo que da por satisfecha la funcionalidad ``\textit{Como usuario, quiero que al final se muestre un ranking de puntos de la partida}''.



\section{RETROSPECTIVA}
\subsection{More Of}
\begin{itemize}
\item Es necesario que los miembros del equipo adquieran más experiencia con el manejo de Git y GitHub, así como los flujos de trabajo para Sistemas de Control de Versiones Descentralizados.

\item Aumentar el grado de comunicación entre los miembros del equipo y la periodicidad de la misma.

\end{itemize}
\subsection{Keep Doing}
\begin{itemize}
\item Trabajar de forma paralela y asignar una tarea (no siempre la misma para tener una visión global del proyecto) a grupos de entre los miembros del equipo ha agilizado la producción de software.

\item Mantener a todos los miembros al tanto de los cambios, modificaciones o direcciones que toma el proyecto a lo largo del proyecto.
\end{itemize}

\subsection{Start Doing}
\begin{itemize}
\item Fijar dos días a la semana en los que podamos reservar un aula de trabajo en grupo para realizar avances del proyecto de forma paralela a los días de clase.

\item Construir un flujo de trabajo definido que se adapte bien al formato distribuido de Git y que permita trabajar de forma paralela entre miembros con distintas funcionalidades.
\end{itemize}

\subsection{Stop Doing}
\begin{itemize}
\item Hacer una planificación deficiente del Sprint en cuanto a definición clara sobre que se va hacer durante el mismo.
\end{itemize}

\subsection{Less of}
\begin{itemize}
\item Es necesario dedicarle más tiempo al diseño de la práctica que a la programación de la misma porque ahorra tiempo y mejora la calidad de trabajo y del software.
\end{itemize}

\section{PLANIFICACIÓN DEL SIGUIENTE SPRINT}
\end{document}