\documentclass{article}
\title{SPRINT 3}
\date{\today}
\author{Grupo PMC}
\usepackage[spanish]{babel}
\usepackage[top=2.5cm, bottom=2.5cm, left=3cm, right=3cm]{geometry}
\clubpenalty=10000 %líneas viudas NO
\widowpenalty=10000 %líneas viudas NO

\begin{document}
\maketitle
\section{REVISIÓN DEL SPRINT}
En primer lugar, la funcionalidad de ``\textit{Como usuario, me gustaría poder jugar a la versión de Rolit por equipos}'' ha sido completada con éxito. La refactorización completa de la parte de la lógica del juego ha permitido que incluir el resto de modos de juego no modifique en absoluto la distribución de responsabilidades que se ha hecho hasta ahora. La inclusión de los Builders ha posibilitado que la creación de los Game quede completamente al margen del resto de clases, que simplemente manejan una abstracción del tipo Game.

La funcionalidad de ``\textit{Como usuario, me gustaría poder jugar a Rolit con una interfaz agradable}'' ha sido completada con éxito en su fase Beta. Nos permite depurar y experimentar con esta distribución de componentes hasta tener un diseño sólido para futuras mejoras.

\section{RETROSPECTIVA}
\subsection{More Of}
\begin{itemize}
\item Es necesario que los miembros del equipo adquieran más experiencia con el manejo de Git y GitHub, así como los flujos de trabajo para Sistemas de Control de Versiones Descentralizados. Se observa una mejora notable con respecto al sprint anterior.

\item Es necesario poner al día los diagramas UML del proyecto.

\end{itemize}
\subsection{Keep Doing}
\begin{itemize}
\item Trabajar de forma paralela y asignar una tarea (no siempre la misma, para tener una visión global del proyecto) a grupos de entre los miembros del equipo ha agilizado la producción de software.

\item Mantener a todos los miembros al tanto de los cambios, modificaciones o direcciones que toma el proyecto a lo largo del proyecto.

\item Mantener el grado de comunicación entre los miembros del equipo y la periodicidad de la misma.

\item Continuar con las reuniones dos días a la semana en los que reservamos un aula de trabajo en grupo para realizar avances del proyecto de forma paralela a los días de clase.

\item Seguir asignando suficiente tiempo para la depuración del código.

\item Fijar con antelación una reunión para llevar a cabo el Sprint Review.

\item Seguir haciendo reuniones de debate sobre cuál es el mejor diseño antes de empezar a programar.
\end{itemize}

\subsection{Start Doing}

\subsection{Stop Doing}
\begin{itemize}
\item Deberíamos dejar de hacer la documentación lo último porque tener archivos de desarrollo facilita que todo el equipo esté al tanto del estado de cierta funcionalidad y lo pueda tener en cuenta en sus diseños.
\end{itemize}

\subsection{Less of}
\begin{itemize}
\item Deberíamos dedicarle algo menos de tiempo al código en favor de la documentación.
\end{itemize}

\section{PLANIFICACIÓN DEL SIGUIENTE SPRINT}
Durante el siguiente Sprint, el equipo de desarrollo llevará a cabo las siguientes tareas de la historia de usuario "Como usuario, me gustaría que Rolit introduzca características pensando en las posibilidades que brinda el multijugador":
\begin{itemize}
\item \textit{Jugar a Rolit en red}.
\item \textit{Jugar a Rolit contra una inteligencia artificial}.
%\end{itemize}
%A nivel de programación, necesitamos cubrir los siguientes objetivos de diseño:
%\begin{itemize}
%\item Aplicar el patrón Modelo-Vista-Controlador al modo consola
%\item Aplicar un buen tratamiento de excepciones (tanto internas como de entrada errónea).
%\item Resolver algunos conflictos de código que se ha quedado obsoleto.
%\item Poner al día GitHub y el UML.
\end{itemize}

\end{document}