\documentclass{article}
\title{HISTORIAS DE USUARIO}
\date{\today}
\author{Grupo PMC}
\usepackage[spanish]{babel}
\usepackage[top=2.5cm, bottom=2.5cm, left=3cm, right=3cm]{geometry}
\clubpenalty=10000 %líneas viudas NO
\widowpenalty=10000 %líneas viudas NO

\begin{document}
\maketitle

\begin{itemize}
\item Como usuario, me gustaría que el número de jugadores fuese variable porque así se adapta a diferentes grupos de personas.

\item Como usuario, me gustaría que se puediesen personalizar los colores con los que jugamos cada jugador porque hace más visual el juego.

\item Como usuario, me gustaría que hubiese tableros distintos para poder tener distintas estrategias a la hora de jugar.
\begin{itemize}
	\item Como usuario, me gustaría que el tamaño del tablero sea variable porque se adapta mejor a los grupos variables.
	\item Como usuario, me gustaría que la forma del tablero fuese variable para que cambiasen las estrategias.
\end{itemize}

\item Como usuario, me gustaría que hubiese varias modalidades de juego sobre rolit:
\begin{itemize}
\item Clásica
\item Por equipos
\item Contra la CPU
\item Online
\end{itemize}

\item Como usuario, me gustaría que se pudiese guardar y cargar partida para continuar más tarde porque permite poner en pausa el juego.

\item Como usuario, me gustaría poder poner mi nombre como nickname durante la partida.

\item Como usuario, quiero que al final se muestre un ranking de puntos de la partida.

\item Como usuario, me gustaría que tenga una interfaz gráfica amable porque hace más fácil jugar.

\item Como usuario, me gustaría que se pudiera guardar repeticiones de partida para poder revisarlas más tarde.

\item Como usuario, me gustaría que hubiese un tutorial para entender bien las normas y excepciones del juego.

\item Como usuario, me gustaría jugar al rolit siguiendo un conjunto mínimo de normas (mirar normas rolit).
\end{itemize}

\section*{SPRINT BACKLOG 1}
\subsubsection*{Como usuario, me gustaría poder jugar a rolit siguiendo un conjunto mínimo de normas (mirar normas rolit)}
\begin{itemize}
\item Consta de un tablero cuadrado de 8x8 casillas
\item Consta de hasta 4 jugadores con colores asignados: amarillo, rojo, verde y azul.
\item Las piezas del juego son cubos redondos de colores que se introducen en las casillas del tablero.
\item Gana el jugador que más bolas de su color tenga sobre el tablero cuando este se llena completamente.
\item El juego acaba cuando el tablero se llena completamente
\item Se decide al azar quien coloca el primer cubo.
\item Solo se puede colocar un cubo redondo en una casilla vacía adyacente a una ya ocupada, excepto el primer cubo que se coloca en la posición que se desee.
\item Un cubo se dice atrapado, cuando se encuentra en la línea que une un cubo recién puesto con otro del mismo color.
\item Cuando se coloca un cubo, se cambian de color todos los cubos atrapados en las direcciones válidas.
\item Las direcciones válidas son las líneas rectas dadas por alguna de las casillas adyacentes.
\item En cada turno un jugador pone un único cubo y se pasa turno al siguiente jugador después de haber cambiado de color los posibles cubos atrapados.
\item El orden de turnos es la secuencia: amarillo, rojo, verde, azul.
\end{itemize}

\subsubsection*{Como usuario, me gustaría poder guardar y cargar partida}
\begin{itemize}
\item Se debe poder guardar la partida y salir del juego en cualquier momento durante la ejecución del mismo.
\item Al iniciar el juego, debe poderse elegir entre cargar una partida guardada o iniciar una nueva.
\item Se pueden tener varias partidas guardadas
\item Para cargar una partida, se muestra una lista para poder elegir la partida a continuar.
\item Para guardar partida, se debe eligir un nombre identificativo para denotar a dicha partida.
\end{itemize}

\newpage
\section*{SPRINT BACKLOG 2}
\subsubsection*{Como usuario, me gustaría que el número de jugadores fuese variable para adaptarse mejor a diferentes grupos de personas.}

\begin{itemize}
\item Cuando el usuario decida crear una nueva partida, se le debe preguntar por el número de jugadores para dicha partida.

\item El número mínimo de jugadores es 2.

\item El número máximo de jugadores es 10.

\end{itemize}

\subsubsection*{Como usuario, me gustaría que se pudiesen personalizar los colores con los que jugamos cada uno para hacer más visual el juego.}
\begin{itemize}

\item Los jugadores deben poder elegir entre los siguientes colores: rojo, amarillo, azul, rosa, verde, morado, negro, naranja, marrón y beige.

\end{itemize}


\subsubsection*{Como usuario, me gustaría poder salir del juego en cualquier momento.}

\begin{itemize}

\item La partida debe finalizar cuando cualquier jugador así lo indique.

\item Se usará un comando exit para llevar a cabo esta funcionalidad.

\end{itemize}

\subsubsection*{Como usuario, me gustaría que hubiese distintos tamaños de tablero seleccionables.}

\begin{itemize}

\item Se puede elegir la longitud del lado del tablero (cuadrado).

\item El tamaño mínimo será de 8 x 8 y el máximo de 15 x 15 casillas.

\end{itemize}


\end{document}